%-*- coding: UTF-8 -*-
% gougu.tex
% 勾股定理
\documentclass[UTF8]{ctexart}

\title{杂谈勾股定理}
\author{张三}
\date{\today}

\bibliographystyle{plain}
\newtheorem{thm}{定理}
\usepackage{graphicx}
\usepackage{float}
\begin{document}

\maketitle
\tableofcontents
\section{勾股定理在古代}
西方称勾股定理为毕达哥拉斯定理,将勾股定理的发现归功于公元前 6 世纪的
毕达哥拉斯学派。该学派得到了一个法则,可以求出可排成直角三角形三边的三
元数组。毕达哥拉斯学派没有书面著作,该定理的严格表述和证明则见于欧几里
德\footnote{欧几里德,约公元前 330——275 年。}《几何原本》的命题 47:“直角三角形斜边上的正方形等于两直角边上的两
个正方形之和。”证明是用面积做的。

我国《周髀算经》载商高(约公元前 12 世纪)答周公问:
\begin{quote}
\zihao{-5}\kaishu 勾广三,股修四,径隅五。
\end{quote}
又载陈子(约公元前 7--6 世纪)答荣方问:
\begin{quote}
\zihao{-5}\kaishu 若求邪至日者,以日下为勾,日高为股,勾股各自乘,并而开方除之,得邪至日。
\end{quote}
都较古希腊更早。后者已经明确道出勾股定理的一般形式。图 1 是我国古代对勾股定理
的一种证明。

\begin{figure}[ht]
  \centering
  \includegraphics[scale=0.6]{test_image.jpg}
  \caption{宋赵爽在《周髀算经》注中作的弦图(仿制),该图给出了勾股定理的一个极具对称美的证明。}
  \label{fig:xiantu}
\end{figure}

\section{勾股定理的近代形式}
勾股定理可以用现代语言表述如下:
\begin{thm}[勾股定理]
直角三角形斜边的平方等于两腰的平方和。

可以用符号语言表述为:设直角三角形 ABC,其中 $\angle C = 90^\circ$,则有
\begin{equation}
  AB^2 = BC^2 + AC^2.
\end{equation}
\end{thm}

满足式(1)的整数称为勾股数。第 1 节所说毕达哥拉斯学派得到的三元数组就是勾股数。
下表列出一些较小的勾股数:
\begin{table}[H]
\begin{tabular}{|rrr|}
\hline
直角边 $a$ & 直角边 $b$ & 斜边 $c$\\
\hline
3 & 4 & 5 \\
5 & 12 & 13 \\
\hline
\end{tabular}%
\qquad
($a^2 + b^2 = c^2$)
\end{table}
\bibliography{math}

\end{document}